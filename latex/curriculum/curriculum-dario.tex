\documentclass[a4paper, 10pt]{article}

\usepackage[italian]{babel}

\begin{document}
\author{Dario Lombardo}

\pagestyle{empty}

\title{Curriculum Vitae}
\date{Ottobre 2009}
\maketitle

\begin{center}
\begin{tabular}{l l}
	\textbf{Nome: } & Dario LOMBARDO\\
	\textbf{Data di nascita: } & 24 novembre 1975\\
	\textbf{Luogo di nascita: } & Torino\\
	\textbf{Indirizzo: } & V. P. Cossa 280/32, \\
				& 10151 Torino\\
	\textbf{Telefono: } & 331/6001292\\
	\textbf{Email: } & dario.lombardo@libero.it\\
	\textbf{Stato civile: } & coniugato\\
	\textbf{Servizio di leva: } & congedato\\
\end{tabular}
\end{center}

\ \\
\ \\
\paragraph{STUDI SVOLTI}
\ \\
\ \\
Ottobre 2009: Rinnovo certificazione CISCO ``Cisco Security Solutions
and Design Specialist''
\\
\ \\
Dicembre 2007: Conseguimento certificazione CISCO ``Cisco Security Solutions
and Design Specialist''
\\
\ \\
Ottobre 2007: Conseguimento certificazione CISCO ``CCDA'' (Cisco Certified
Design Associate)
\\
\ \\
Luglio 2004: Conseguimento della certificazione ``CEH'' (Certified Ethical
Hacker) sulle tecniche di intrusione informatica, rilasciato da EC-Council.
\\
\ \\
Febbraio 2003: Conseguimento Master in telecomunicazioni rilasciato da
COREP (Consorzio per la ricerca e l'educazione permanente) e da TILS
(Telecom Italia Learning Services).
\\
\ \\
Settembre 2002: Corso CISCO ``Building Scalable Cisco Internetworks'' presso
PRES S.r.l.
\\
\ \\
Marzo 2001: Conseguimento Laurea in Ingegneria Informatica presso il
Politecnico di Torino con la votazione di
99/110, con tesi di laurea dal titolo ``Gestione delle politiche di sicurezza in reti TCP/IP:
implementazione di IPSP (IP Security Policy)''. 
Relatori: Prof. \mbox{Antonio Lioy}, Prof. \mbox{Marco Mezzalama}.
\\
\ \\
1994: Diploma di maturit\`a scientifica presso l'istituto A. Volta di Torino
con la votazione di 60/60.
\\
\paragraph{ESPERIENZE}
\ \\
\ \\
2009 - oggi: Telecom Italia IT, responsabile di progetto sul tema sicurezza del DNS: 
progettazione e sviluppo di un prototipo per la protezione in rete del DNS basato
su SDK Juniper.
\\
\ \\
2008 - oggi: Telecom Italia IT, responsabile di attivit\`a sul tema sicurezza 
delle applicazioni: sperimentazione di soluzioni della suite Fortify per la produzione
di codice sorgente sicuro, sperimentazione di soluzioni di Web assessment.
\\
\ \\
2006 - 2008: Telecom Italia IT, responsabile di progetto sul tema sicurezza del 
VoIP: progettazione, sviluppo, test e messa in campo di un prototipo per il 
monitoraggio del servizio VoIP (raccolta dati, engine di detection, interfaccia 
utente).
\\
\ \\
2003 - 2006: Telecom Italia Lab, responsabile di attivit\`a sul tema sicurezza 
del routing IP: progettazione, sviluppo, test e messa in campo di un prototipo 
per il monitoraggio dei principali protocolli di routing (OSPF, BGP), comprensivo
del sistema di collezione e dell'interfaccia utente (soluzione oggi in esercizio
nella rete Telecom Italia).
\\
\ \\
2006: responsabile corso di formazione di security per personale tecnico della 
funzione tecnica Technical Security.
\\
\ \\
2006: responsabile corso di formazione di security per personale tecnico TILAB.
\\ 
\ \\
2005: responsabile corso di formazione di security per personale tecnico TILAB.
\\
\ \\ 
2004: resposabile corso di formazione di security per personale progettazione 
Top Client Telecom Italia.
\\ 
\ \\
2003: docente corso di formazione di security per personale tecnico TILAB.
\\
\ \\
2002: docente corso di formazione di security per personale tecnico Telecom Italia.
\\
\ \\
2001 - 2003: Telecom Italia Lab, gruppo di Vulnerability Assessment: attivit\`a di 
analisi di sicurezza verso sistemi interni all'azienda e verso clienti esterni, 
con attivit\`a sia effettuate via rete che presso la sede dei clienti.
\\
\ \\
1999-2000: Attivit\`a di borsista presso il laboratorio CCLINF, Politecnico di Torino.
\\
\paragraph{LINGUE STRANIERE}
\ \\
\ \\
INGLESE: livello \emph{C1} (classificazione ``Common European Framework of
Reference for Languages'')
\\
\paragraph{CONOSCENZE TECNICHE}
\ \\
\ \\
Elevata conoscenza dei protocolli di rete e delle problematiche di sicurezza
ad essi connesse. La conoscenza \`e particolarmente approfondita per i
protocolli della suite TCP/IP e dei suoi protocolli di routing, al quale \`e 
stato dato ampio spazio di approfondimento con l'intento di avere una
visione il pi\`u ampia possibile su ogni aspetto di sicurezza. L'esperienza
\`e stata particolarmente approfondita grazie alla
partecipazione ai principali convegni mondiali di security ed all'incontro
con i massimi esperti del settore.\\
Ampia esperienza di progettazione e sviluppo software in team (software
engineering) particolarmente rivolta agli ambienti adatti allo sviluppo
prototipale (Linux/Unix e C/C++).\\
Esperienza di sviluppo di applicazioni web utilizzando metodologie agili in 
ambiente RubyOnRails in cui la velocit\`a di sviluppo e l'orientamento al
risultato sono il focus principale.
\\
\ \\
\textbf{Corsi universitari seguiti}: Prof. \mbox{Antonio Lioy}: 
\emph{Sistemi di elaborazione 
distribuiti}; Prof. \mbox{Marco Mezzalama}: \emph{Sistemi di elaborazione I}; 
Prof. \mbox{Silvano Gai}: \emph{Reti di calcolatori II}; Prof. \mbox{Angelo
R. Meo}: \emph{Sistemi di elaborazione II}.
\\
\paragraph{PARTECIPAZIONI A CONVEGNI}
\ \\
\ \\
LinuxDay (Torino): nel 2010 (speaker)
\\
CanSecWest (Vancouver): nel 2010
\\
ICIN (Bordeaux): nel 2009 (speaker)
\\
Black Hat Europe (Amsterdam): nel 2003, 2004, 2005
\\
Black Hat USA (Las Vegas): nel 2006, 2007, 2008
\\
Defcon (Las Vegas): nel 2005, 2006, 2007, 2008
\\
Cisco Networkers Europe (Cannes): nel 2004, 2005
\\
\paragraph{PUBBLICAZIONI}
\ \\
\ \\
2009: \mbox{P. De Lutiis, D. Lombardo}, ``An innovative way to analyze large ISP 
data for IMS security and monitoring'', ICIN2009: International conference on 
Intelligent Networks, October 26-29/2009, ISBN: 978-1-4244-4693-3, pages 1-6, 
Digital Object Identifier: 10.1109/ICIN.2009.5357065, Current Version Published: 
2009-12-22 
\\
\ \\
2009: \mbox{P. De Lutiis, D. Lombardo}, ``Sicurezza delle reti VoIP: tecniche di 
anomaly detection'', Notiziario Tecnico Telecom Italia, anno 18, numero 2-2009, pp.
4-24.
\\
\ \\
2007: \mbox{M. Baltatu, S. Di Paola, D. Lombardo}, brevetto dal titolo ``Anomaly
detection for link-state routing protocols'', p.n. WO2009084053 (A1), 
sull'utilizzo di \emph{Support Vector Machine (SVM)} nell'anomaly detection su
reti IP.
\\
\ \\
2001: \mbox{M. Baltatu, A. Lioy, D. Lombardo, D. Mazzocchi}, ``Towards a policy 
system for IPsec: issues and an experimental implementation'', ICONN-2001: IEEE 
International Conference on Networks 2001, Bangkok (Thailand), October 10-12, 
2001, pp. 146-151 ISBN 0-7695-1186-4 
\\
\ \\
2000: \mbox{S. Di Paola, D. Lombardo}, ``La sicurezza del sistema operativo 
Linux, (securing Red Hat 5.2)'' nell'ambito del progetto ILDP (\emph{Italian 
Linux Documentation Project}).
\\

\end{document}
